\documentclass[10pt]{article}
\usepackage[final]{pdfpages}
\usepackage{cleveref}
\usepackage{xparse}
\usepackage{hyperref}
\usepackage{geometry}
\usepackage{amsmath}
\usepackage{graphicx}
\usepackage{caption}
\usepackage{subcaption}
\usepackage[section]{placeins}
\usepackage{listings}
\usepackage{verbatim}
%\usepackage{harvard}
%\usepackage[dcucite,abbr]{harvard}
%\usepackage{sectsty}
%\usepackage{html}
%\usepackage{url}
%\sectionfont{\rmfamily\mdseries\Large}
%\subsectionfont{\rmfamily\mdseries\itshape\large}

\geometry{
  %body={6.5in, 8.5in},
  left=2.5cm,
  right=2cm,
  top=2cm,
  bottom=2cm
}

\linespread{1.213}
\begin{document}
\includepdf{./BossPuzzle.pdf}
\newpage
indeks
\newpage
\section{Introduction}
This report is about the Nine-Puzzle.  The document will give a little bit of background information and also where did the nine-puzzle start.  This report will also include how to play the game. A step by step guide is included that tell you how to set up the puzzle and how to run the Nine-puzzle game that is included in the folder.
\section{Background and History}


\section{Literature Study}

\section{User Guide}
\subsection{How to make a .csv file}
Step 1:
\\Open the file named waarde\_puzzle.csv in notepad$++$.
Remember that the file must have \,in between the numbers and there must only be 8 numbers and a lowercase b where the blank space in the puzzle must be.Figure~\ref{csv}
An example of what the .csv file looks in notepad$++$. The first row is the puzzle that is presented to the player and the second row is the finale puzzle Figure~\ref{waardes}.  If the player has saved its puzzle to continue later there will be a third row in the .csv file that shows the moves that the player has done already.
\begin{figure}
\centering
\includegraphics[scale=0.7]{./Prente/csv.png}
\caption{}
\label{csv}
\end{figure}
\begin{figure}
\centering
\includegraphics[scale=1]{./Prente/waardes.png}
\caption{}
\label{waardes}
\end{figure}
\subsection{How to begin the Nine-puzzle}
Step 1:
\\Right click on the execute.bat file and open it with notepad.
Change the name of the .csv file.  Save the file and exit it.  Double click on the file to run the file. Figure~\ref{csv}
\\Step 2:
\begin{figure*}[b!]
    \centering
    \begin{subfigure}[b]{0.5\textwidth}
        \centering
        \includegraphics[scale=0.8]{./Prente/begincmd.png}
        \caption{if you have run the execute.bat file}
        \label{begincmd}
    \end{subfigure}%
 
    \begin{subfigure}[b]{0.5\textwidth}
        \centering
        \includegraphics[scale=0.8]{./Prente/MessageBegin.png}
        \caption{The program will show this message directly after you have run the execute.bat file}
        \label{MessageBegin}
    \end{subfigure}
    \caption{\label{1}}
   \end{figure*}
If the program has run you will see figure Figure~\ref{1}
\\Step 3: Enter a y for you are a new player or n for you are not a new player Figure~\ref{prent1}
\begin{figure}
\centering
 \includegraphics[scale=0.8]{./Prente/prent1.png}
  \caption{This figure asks if you are a new player or not}
  \label{prent1}
\end{figure}
\\Step 4:
\begin{figure}
\centering
 \includegraphics[scale=0.8]{./Prente/prent2.png}
 \caption{}
 \label{prent2}
\end{figure}
After step 3 is completed the begin puzzle will be given. Figure~\ref{prent2}
\\Step 5:
\begin{figure}
\centering
 \includegraphics[scale=0.8]{./Prente/prent3.png}
 \caption{}
 \label{prent3}
\end{figure}
Enter the number that you want to move like in Figure~\ref{prent3}. The game will move that number to the blank space(b) and give the updated puzzle again.
\subsection{Functionality}
How to stop save and exit the program:
If you want to stop playing and play again later then you can type 0 then the puzzle will be saved and the game will exit. Figure~\ref{2}
\begin{figure*}[b!]
    \centering
    \begin{subfigure}[b]{0.5\textwidth}
        \centering
        \includegraphics[scale=0.8]{./Prente/prent4.png}
        \caption{If you type in 0}
        \label{prent4}
    \end{subfigure}%
 
    \begin{subfigure}[b]{0.5\textwidth}
        \centering
        \includegraphics[scale=0.8]{./Prente/prent5.png}
        \caption{The message will show that you work has been saved}
        \label{prent5}
    \end{subfigure}
    \caption{\label{2}}
   \end{figure*}
   
If you have won the game the program will save your work and show a message to congratulate you and show how many moves you have made to solve the puzzle. Figure~\ref{prent6} 
\begin{figure}
\centering
\includegraphics[scale=0.8]{./Prente/prent6.png}
\caption{}
\label{prent6}
\end{figure}

If you want to view the finale puzzle you type in 99. Figure~\ref{prent7}
\begin{figure}
\centering
\includegraphics[scale=0.8]{./Prente/prent7.png}
\caption{}
\label{prent7}
\end{figure}

\section{Code}
 \begin{tiny}
 \lstinputlisting[language=java, firstline=1, lastline=222]{./NinePuzzle.java}
   \end{tiny}
   \section{Bibliography}

\end{document}